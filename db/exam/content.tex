\section{Экзамен}

\subsection{Базы данных и системы управления базами данных. Определения, основные функции и классификация}

\textbf{База данных} --- совокупность хранимых операционных данных, используемых прикладными системами некоторого предприятия.

\textbf{Операционные данные} --- быстрые данные (чтобы их можно было быстро прочитать).

Можно читать, писать, обновлять

Данные делятся на две большие категории

\begin{enumerate}
	\item Читать быстро. 
	\item Быстро писать.
\end{enumerate}

\textbf{OLAP} --- online analtyc proccesing (операциооные данные).

\textbf{OLTP} --- online transaction proccessing (перманентые данные).

\begin{table}[ht!]
	\begin{center}
		\caption{}
		\label{tbl:best}
		\begin{tabular}{|c|c|}
			\hline
			OLAP & OLTP \\
			\hline
			чтение & вставка, удаление, обновление \\
			\hline
			минимальное время отклика & минимальное время вставки, удаление, обновление \\
			\hline
		\end{tabular}
	\end{center}
\end{table}

\textbf{Транзакции} --- либо все действия, либо никакие действия.

\textbf{База данных} --- это самодокументирования собрание интегрированных записей.

\begin{enumerate}
	\item Самодументированное --- из нее понятно, что в ней хранится. Понятно какие объекты и принципы действий. -> журналы.
	\item Запись --- это события которые надо где-то хранить.
	\item Интегрированных --- записи которые имеют некоторую структуру и имеют некоторую структуру.
\end{enumerate}

\textbf{Любая бд хранит}:

\begin{enumerate}
	\item метаданные;
	\item файлы данных,
	\item Индексы (indexes), которые представляют связи между данными,  a также служат для повышения производительности приложений базы данных.
	\item Может содержать метаданные приложений (application metadata).
\end{enumerate}

\textbf{Основные храктеристики, требования}

\begin{enumerate}
	\item \textbf{Неизбыточность данных} --- каждое данное присутствует в БД в единственном экземпляре.
	\item \textbf{Совместное использование данных} многими пользователями.
	\item \textbf{Эффективность доступа} к БД - высокое быстродействие, т. е. малое время отклика на запрос.
	\item \textbf{Целостность данных} --- соответствие имеющейся в БД информации её внутренней логике, структуре и всем явно заданным правилам.
	\item \textbf{Безопасность данных} --- защита данных от преднамеренного или непреднамеренного искажения или разрушения данных.
	\item \textbf{Восстановление данных} после программных и аппаратных сбоев.
	\item \textbf{Независимость данных} от прикладных программ.
\end{enumerate}

\textit{Система управления базами данных} (СУБД) --- приложение, обеспечивающее создание, хранение, обновление и поиск информации в базах данных.

\textbf{Зачем СУБД}

\begin{enumerate}
	\item Управление данными во внешней памяти.
	\item Управление буферами оперативной памяти.
	\item Управление транзакциями.
	\item Журнализация.
	\item Поддержка языка или языкового пакета (-ов).
\end{enumerate}

\textbf{Классификация СУБД}

\begin{enumerate}
	\item Дореляционные
		\begin{itemize}
			\item Инвертированные списки (файлы) (Как дерево)
			\item Иерархичекие (что-то похожее на хэш)
			\item Сетевые
		\end{itemize}
	\item Реляционные
	\item Постреляционные
\end{enumerate}

\textbf{Архитектура хранения данных}

\begin{enumerate}
	\item Локальные.
	\item Распределенные.
	\item По способу обращения к данным.
	\begin{itemize}
		\item Файл серверные.
		\item Клиент серверные (PostGress, MSSQL, Oracle, MySQL, Mongo).
		\item Встраиваемые (SQLlite).
		\item Сервисно-ориентированные (KafcaBD).
		\item Прочее - time series.
	\end{itemize}
\end{enumerate}

\subsection{Семантическое моделирование данных}

\textbf{Определение}: Дискретная


\subsection{Реляционная модель данных: структурная, целостная, манипуляционная части. Реляционная алгебра. Исчисление кортежей}

\textbf{Свойства}


\subsection{Теория проектирования реляционных баз данных: функциональные зависимости, нормальные формы}

Пусть 


\subsection{Теория проектирования хранилищ данных. Основные принципы построения. ETL и ELT процессы}

\textbf{Определение}

\subsection{Транзакции. Определение, свойства и уровни изоляции транзакций. Неблагоприятные эффекты, вызванные параллельным выполнением транзакци , и способы их устранения. Управление транзакциями и способы обработки ошибок}

\textbf{Свойства} для n = 2


\subsection{Блокировки. Определение, свойства, иерархии, гранулярность и взаимоблокировки, алгоритмы обнаружения взаимоблокировок}

\textbf{Определение}


\subsection{Журнализация. Операции журнала транзакций и его логическая и физическая архитектуры. Модели восстановления. Метаданные}

\textbf{Нет вопроса}

\subsection{Безопасность и Аудит. Ключевые понятия и участники системы безопасности. Модели управления доступом}

\textbf{Этого вопроса нет}


\subsection{MPP системы. Распределенное и колоночное хранение. Распределенные вычисления, модель MapReduce. Обеспечение отказоустойчивости.}

Пусть

\subsection{In-Memory базы данных. Преимущества и недостатки. Примеры использования}

Учитывая равенство $P\{Y < y\} = 	F_Y(y)$, приходим к формуле \ref{jopa2}.

\subsection{Инструкции языка описания данных, инструкции языка обработки данных, инструкции безопасности, инструкции управления транзакциями}

Когда $X_1, X_2$ являются \textit{независимыми случайными величинами, то есть их двумерная плотность распределения}



\subsection{Объекты базы данных: функции, процедуры, триггеры и курсоры}


\subsection{Оптимизация запроса: индексы, партиционирование, сегментирование}

\textbf{Дисперсией} случайной величины $X$ называют число



\subsection{План запроса. Этапы выполнения запроса}

Пусть $X$ --- случайная величина.
