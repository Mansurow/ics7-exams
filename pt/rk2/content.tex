\section{Рубежный контроль 2}

\subsection{Сформулировать определение несовместных событий. Как связаны свойства несовместности и независимости событий?}

\textbf{Определение}. 

События $A$ и $B$ называются несовместными, если их произведение пусто. В противном случае события $A$ и $B$ называются совместными.

\textbf{Определение}. 

События $A$ и $B$ называются независимыми, если $P(AB) = P(A) \cdot P(B)$.

\textbf{Как связаны}

Если события несовместные, то они не могут быть независимыми.

\subsection{Сформулировать геометрическое определение вероятности.}

Геометрическое определение вероятности является обобщением классического определения на случай, когда $|\Omega| = \infty$. 

\textbf{Пусть}

\begin{enumerate}
	\item $|\Omega|$ $\subseteq$ $R^n$;
	\item $\mu(\Omega|$ < $\infty$, где $\mu$ --- некая мера. Если $n$ = 1, то $\mu$ --- это длина; если n = 2, то $\mu$ --- площадь; если n = 3 --- объём. Можно определить меры и при больших n; 
	\item Возможность принадлежности некоторого элементарного исхода случайного эксперимента событию $ A \subseteq$ $\Omega$ пропорциональна мере этого события и не зависит от формы события A и его расположения внутри $\Omega$. 
	
\end{enumerate} 

Тогда \textbf{Определение}. 

Вероятностью случайного события $ A \subseteq$ $\Omega$ называют число

\begin{equation}
	P(A) = \frac{\mu(A)}{\mu(\Omega)}
\end{equation} 

\subsection{Сформулировать определение сигма-алгебры событий. Сформулировать ее основные свойства.}

Для строгого аксиоматического определения вероятности необходимо уточнить понятие события:

\begin{enumerate}
	\item Данное выше определение события как произвольного подмножества множества $\Omega$ в случае бесконечного множества $\Omega$ приводит к противоречивой теории (см. парадокс Рассела);
	\item Таким образом, необходимо в качестве события рассматривать не все возможные подмножества множества $\Omega$, а лишь некоторые из них; 
	\item Набор подмножеств множества $\Omega$, выбранных в качестве событий, должен обладать рядом свойств. Понятно, что если A и B --- связанные со случайным экспериментом события и известно, что в результате эксперимента они произошли (или не произошли), то естественно знать, произошли ли события $A + b, A \cdot B, \dots$
\end{enumerate}

Эти соображения приводят к следующему определению.

\textbf{Пусть} 

1. $\Omega$ --- пространство элементарных исходов, связанных с некоторым случайным экспериментом; 
2. $\beta \neq \varnothing$ --- система (набор) подмножеств в множестве $\Omega$.

\textbf{Определение}. 

$\beta$ называется сигма-алгеброй событий, если выполнены условия: 
\begin{enumerate}
	\item если $A \in \beta$, то $\overline{A} \in \beta$;
	\item если $A_1, \dots, A_n, \dots \in \beta$, то $A_1 + \dots + A_n + \dots \in \beta$
\end{enumerate}

\textbf{Свойства}

\begin{enumerate}
	
	\item $\Omega \in \beta$;
	\item $\varnothing \in \beta$;
	\item если $A_1, \dots A_n, \dots \in \beta$, то $A_1 \cdot A_2 \cdot \dots A_n \dots \in \beta$;
	\item если $A_1, A_2 \in \beta$, то $A_1 \backslash A_2 \in \beta$
\end{enumerate}


\subsection{Сформулировать аксиоматическое определение вероятности. Сформулировать основные свойства вероятности.}

\textbf{Пусть} 

\begin{enumerate}
	\item $\Omega$ --- пространство элементарных исходов случайного эксперимента;
	\item $\beta$ --- сигма-алгебра, заданная на $\Omega$.
\end{enumerate}

\textbf{Определение}

Вероятностью (вероятностной мерой) называется функция $P: \beta \rightarrow R$

\begin{enumerate}
	\item $\forall A \in \beta => P(A) \geqslant 0$; (аксиома неотрицательности);
	\item $P(\Omega) = 1$ (аксиома нормированности);
	\item Если $A_1, \dots, A_n, \dots$ --- попарно несовместные события, то вероятность осуществления их суммы равна сумме вероятностей осуществления каждого из них по отдельности: $P(A_1 + \dots + A_n + \dots) = P(A_1) + \dots + P(A_n) + \dots$ (расширенная аксиома сложения). 
\end{enumerate}

\begin{enumerate}
	\item $P(\overline{A}) = 1 - P(A)$;
	\item $P(\varnothing) = 0$;
	\item Если $A \subseteq B$, то $P(A) \leqslant P(B)$;
	\item $\forall A \in \beta: 0 \leqslant P(a) \leqslant 1$;
	\item $P(A + B) = P(A) + P(B) - P(AB)$, где A, B $\in \beta$;
	\item Для любого конечного набора событий $A_1, \dots, A_n$ верно
	\begin{equation}
		P(A_1 + \dots + A_n) = + \sum\limits_{1 \leqslant i_1 \leqslant n}P(A_{i_1}) - \sum\limits_{1 \leqslant i_1 \leqslant i_2 \leqslant n}P(A_{i_1}, A_{i_2}) + \sum\limits_{1 \leqslant i_1 \leqslant i_2 \leqslant i_3 \leqslant n}P(A_{i_1}, A_{i_2}, A_{i_3}) - \dots
	\end{equation}
\end{enumerate}

\subsection{Записать аксиому сложения вероятностей, расширенную аксиому сложения вероятностей и аксиому непрерывности вероятности. Как они связаны между собой?}

\textbf{Аксиома сложения}

Сложение --- для $\forall$ конечного набора попарно несовместных событий $A_1, \dots, A_n$ вероятность осуществления их суммы равна сумме вероятностей каждого из них по отдельности: $P(A_1 + \dots + A_n) = P(A_1) + \dots + P(A_n)$.

\textbf{Расширенная Аксиома сложения}

Если $A_1, \dots, A_n, \dots$ --- попарно несовместные события, то вероятность осуществления их суммы равна сумме вероятностей осуществления каждого из них по отдельности: $P(A_1 + \dots + A_n + \dots) = P(A_1) + \dots + P(A_n) + \dots$.

\textbf{Непрерывность}

Для любой неубывающей последовательности событий $A_1 \subseteq A_2 \subseteq \dots \subseteq A_n \subseteq \dots$ и события $ A = \bigcup\limits_i A_i$ верно 
\begin{equation}
	P(A) = \lim\limits_{i \rightarrow \infty} P(A_i).
\end{equation}

\textbf{Связанность}

Из аксиомы сложения и непрерывности следует расширенная аксиома сложения.

\subsection{Сформулировать определение условной вероятности и ее основные свойства.}

\textbf{Пусть} 

\begin{enumerate}
	\item $A$ и $B$ --- два события, связанные с одним случайным экспериментом;
	\item дополнительно известно, что в результате произошло событие B и P(B) > 0.
\end{enumerate}

Условной вероятностью осуществления события A при условии, что произошло B, называется число

\begin{equation}
	P(A|B) = \frac{P(AB)}{P(B)}, P(B) \neq 0.
\end{equation}

\textbf{Свойства}

\begin{enumerate}
	\item P(A|B) $\subseteq$ 0;
	\item $P(\Omega|B) = 1$;
	\item $P(A_1 + \dots + A_n + \dots | B) = P(A_1|B) + \dots + P(A_n|B) + \dots$.
\end{enumerate}

\subsection{Сформулировать теоремы о формулах умножения вероятностей для двух событий и для произвольного числа событий.}

\textbf{Теорема}. Формула умножения вероятностей для двух событий 

\textbf{Пусть}

\begin{enumerate}
	\item A, B --- события;
	\item P(A) > 0.
\end{enumerate}

Тогда P(AB) = P(A) P(B|A)

\textbf{Теорема} Формула умножения вероятностей для n событий

\begin{enumerate}
	\item $A_1, \dots, A_n$ --- события;
	\item $P(A_1 \cdot \dots \cdot > 0)$.
\end{enumerate}

Тогда 

\begin{equation}
	P(A_1 \cdot A_2 \cdot \dots \cdot A_n)  = P(A_1) P(A_2 | A_1)P(A_3|A_1A_2) \cdot \dots \cdot P(A_n|A_1) \dots   A_{n-1}).
\end{equation}

\subsection{Сформулировать определение пары независимых событий. Как независимость двух событий связана с условными вероятностями их осуществления?}

Пусть 𝐴 и 𝐵 — два события, связанные с некоторым случайным экспериментом.
Определение. События 𝐴 и 𝐵 называются независимыми, если 𝑃(𝐴𝐵) = 𝑃(𝐴) 𝑃(𝐵).

Замечание. Разумеется, в качестве определения независимых событий логично было бы использовать условия 𝑃(𝐴 | 𝐵) = 𝑃(𝐴) или 𝑃(𝐵 | 𝐴) = 𝑃(𝐵) (6) Однако эти условия имеют смысл лишь тогда, когда 𝑃(𝐴) или 𝑃(𝐵) отлично от нуля. Условие же 𝑃(𝐴𝐵) = 𝑃(𝐴)𝑃(𝐵) «работает» всегда без ограничений.

\subsection{Сформулировать определение попарно независимых событий и событий, независимых в совокупности. Как эти свойства связаны между собой?}

Определение. События 𝐴1, . . . , 𝐴𝑛 называется попарно независимыми, если
∀∀ 𝑖 ̸= 𝑗; 𝑖, 𝑗 ∈ {1, . . . , 𝑛} 𝑃{𝐴𝑖𝐴𝑗} = 𝑃{𝐴𝑖}𝑃{𝐴𝑗}
Определение. События 𝐴1, . . . , 𝐴𝑛 называются независимыми в совокупности, если 
∀ 𝑘 ∈ {2, . . . , 𝑛} ∀∀ 𝑖1 < 𝑖2 < . . . < 𝑖𝑘 𝑃{𝐴𝑖1 , . . . , 𝐴𝑖𝑘 } = 𝑃{𝐴𝑖1 } · . . . · 𝑃{𝐴𝑖𝑘 }
1 <- 2


\subsection{Сформулировать определение полной группы событий. Верно ли, что некоторые события из полной группы могут быть независимыми?}

Пусть Ω — пространство элементарных исходов, связанных с некоторым случайным экспериментом, а (Ω, 𝛽, 𝑃) — вероятностное пространство этого случайного эксперимента. Определение. Говорят,  что события 𝐻1, . . . , 𝐻𝑛 ∈ 𝛽 образуют полную группу событий, если
1. 𝑃(𝐻𝑖) > 0, 𝑖 = 1, 𝑛; 
2. 𝐻𝑖𝐻𝑗 = ∅ при 𝑖 ̸= 𝑗; 
3. 𝐻1 + . . . + 𝐻𝑛 = Ω.
Да, верно.

\subsection{Сформулировать теорему о формуле полной вероятности.}

Теорема. Формула полной вероятности. Пусть 
1. 𝐻1, . . . , 𝐻𝑛 — полная группа событий;
2. 𝐴 ∈ 𝛽 — событие.
Тогда (это выражение называется формулой полной вероятности): 𝑃(𝐴) = 𝑃(𝐴 | 𝐻1)𝑃(𝐻1) + . . . + 𝑃(𝐴 | 𝐻𝑛)𝑃(𝐻𝑛)

\subsection{Сформулировать теорему о формуле Байеса.}

Теорема. Пусть 
1. 𝐻1, . . . , 𝐻𝑛 — полная группа событий; 
2. 𝑃(𝐴) > 0. 

\subsection{Дать определение схемы испытаний Бернулли. Записать формулу для вычисления вероятности осуществления ровно k успехов в серии из n испытаний.}

Определение. 

\subsection{Сформулировать определение элементарного исхода случайного эксперимента и пространства элементарных исходов. Сформулировать классическое определение вероятности. Привести пример.}

Определение. Множество Ω всех исходов данного случайного эксперимента называют пространством элементарных исходов. 
1. Каждый элементарный исход является «неделимым», т. е. он не может быть разбит на более «мелкие» исходы; 
2. В результате каждого эксперимента обязательно имеет место ровно один из входящих в Ω элементарных исходов.

\subsection{Сформулировать классическое определение вероятности. Опираясь на него, доказать основные свойства вероятности}

Свойства вероятности:
1. Вероятность 𝑃(𝐴) > 0 (неотрицательна). 
2. 𝑃(Ω) = 1. 
3. Если 𝐴, 𝐵 — несовместные события, то 𝑃(𝐴 + 𝐵) = 𝑃(𝐴) + 𝑃(𝐵)
Доказательство:

\subsection{Сформулировать статистическое определение вероятности. Указать его основные недостатки.}

1. Некоторый случайный эксперимент произведён 𝑛 раз; 
2. При этом некоторое наблюдаемое в этом эксперименте событие 𝐴 произошло 𝑛𝐴 раз.

У статистического определения полным-полно недостатков:
1. Никакой эксперимент не может быть произведён бесконечное много раз;
2. С точки зрения современной математики статистическое определение является архаизмом, т. к. не даёт достаточно базы для дальнейшего построения теории.

\subsection{Сформулировать определение сигма-алгебры событий. Доказать ее основные свойства.}

Для строгого аксиоматического определения вероятности необходимо уточнить понятие события:
1. Данное выше определение события как произвольного подмножества множества Ω в случае бесконечного множества Ω приводит к противоречивой теории (см. парадокс Рассела); 
2. Таким образом, необходимо в качестве события рассматривать не все возможные подмножества множества Ω, а лишь некоторые из них;
3. Набор подмножеств множества Ω, выбранных в качестве событий, должен обладать рядом свойств. Понятно, что если 𝐴 и 𝐵 — связанные со случайным экспериментом события и известно, что в результате эксперимента они произошли (или не произошли), то естественно знать, произошли ли события 𝐴 + 𝐵, 𝐴 · 𝐵, -𝐴,  . . .
Эти соображения приводят к следующему определению. Пусть 
1. Ω — пространство элементарных исходов, связанных с некоторым случайным экспериментом; 
2. 𝛽 ̸= ∅ — система (набор) подмножеств в множестве Ω.
Определение. 𝛽 называется сигма-алгеброй событий, если выполнены условия: 
1. Если 𝐴 ∈ 𝛽, то 𝐴 ∈ 𝛽; 
2. Если 𝐴1, . . . , 𝐴𝑛, . . . ∈ 𝛽, то 𝐴1 + . . . + 𝐴𝑛 + . . . ∈ 𝛽
Свойства:
1. Ω ∈ 𝛽; 
2. ∅ ∈ 𝛽; 
3. Если 𝐴1, . . . , 𝐴𝑛, . . . ∈ 𝛽, то 𝐴1 · . . . · 𝐴𝑛 · . . . ∈ 𝛽; 
4. Если 𝐴, 𝐵 ∈ 𝛽; то 𝐴 r 𝐵 ∈ 𝛽
Доказательства:

\subsection{Сформулировать аксиоматическое определение вероятности. Доказать свойства вероятности для дополнения события, для невозможного события, для следствия события.}

усть 
1. Ω — пространство элементарных исходов некоторого случайного эксперимента;
2. 𝛽 — сигма-алгебра, заданная на Ω.
Определение. Вероятностью (вероятностной мерой) называется функция
𝑃 : 𝛽 → R
1. ∀𝐴 ∈ 𝛽 =⇒ 𝑃(𝐴) > 0 (аксиома неотрицательности); 
2. 𝑃(Ω) = 1 (аксиома нормированности); 
3. Если 𝐴1, . . . , 𝐴𝑛, . . . — попарно несовместные события, то вероятность осуществления их суммы равна сумме вероятностей осуществления каждого из них по отдельности: 𝑃(𝐴1 + . . . + 𝐴𝑛 + . . .) = 𝑃(𝐴1) + . . . + 𝑃(𝐴𝑛) + . . . (расширенная аксиома сложения).



Доказательства:

\subsection{Сформулировать аксиоматическое определение вероятности. Сформулировать свойства вероятности для суммы двух событий и для суммы произвольного числа событий. Доказать первое из этих свойств.}

Пусть 
1. Ω — пространство элементарных исходов некоторого случайного эксперимента;
2. 𝛽 — сигма-алгебра, заданная на Ω.
Определение. Вероятностью (вероятностной мерой) называется функция
𝑃 : 𝛽 → R
1. ∀𝐴 ∈ 𝛽 =⇒ 𝑃(𝐴) > 0 (аксиома неотрицательности); 
2. 𝑃(Ω) = 1 (аксиома нормированности); 
3. Если 𝐴1, . . . , 𝐴𝑛, . . . — попарно несовместные события, то вероятность осуществления их суммы равна сумме вероятностей осуществления каждого из них по отдельности: 𝑃(𝐴1 + . . . + 𝐴𝑛 + . . .) = 𝑃(𝐴1) + . . . + 𝑃(𝐴𝑛) + . . . (расширенная аксиома сложения).

Формулировка:

\subsection{Сформулировать определение условной вероятности. Доказать, что она удовлетворяет трем основным свойствам безусловной вероятности.}

Пусть
1. 𝐴 и 𝐵 — два события, связанные с одним случайным экспериментом;
2. Дополнительно известно, что в результате эксперимента произошло событие 𝐵.
Определение. Условной вероятностью осуществления события 𝐴 при условии, что произошло 𝐵, называется число

Теорема:
Пусть 
1. Зафиксировано событие 𝐵, 𝑃(𝐵) ̸= 0; 
2. 𝑃(𝐴 | 𝐵) рассматривается как функция события 𝐴. 
Тогда 𝑃(𝐴 | 𝐵) обладает всеми свойствами безусловной вероятности.
Доказательство:

\subsection{Доказать теоремы о формулах умножения вероятностей для двух событий и для произвольного числа событий.}
\subsection{Сформулировать определение пары независимых событий. Сформулировать и доказать теорему о связи независимости двух событий с условными вероятностями их осуществления.}

Пусть 𝐴 и 𝐵 — два события, связанные с некоторым случайным экспериментом. 
Определение. События 𝐴 и 𝐵 называются независимыми, если 𝑃(𝐴𝐵) = 𝑃(𝐴) 𝑃(𝐵).
Теорема. . . . 
1. Пусть 𝑃(𝐵) > 0. Утверждение «𝐴 и 𝐵 — независимы» равносильно 𝑃(𝐴 | 𝐵) = 𝑃(𝐴); 
2. Пусть 𝑃(𝐴) > 0. Утверждение «𝐴 и 𝐵 — независимы» равносильно 𝑃(𝐵 | 𝐴) = 𝑃(𝐵).

\subsection{Сформулировать определение попарно независимых событий и событий, независимых в совокупности. Показать на примере, что из первого не следует второе.}

Определение. События 𝐴1, . . . , 𝐴𝑛 называется попарно независимыми, если
∀∀ 𝑖 ̸= 𝑗; 𝑖, 𝑗 ∈ {1, . . . , 𝑛} 𝑃{𝐴𝑖𝐴𝑗} = 𝑃{𝐴𝑖}𝑃{𝐴𝑗}
Определение. События 𝐴1, . . . , 𝐴𝑛 называются независимыми в совокупности, если
∀ 𝑘 ∈ {2, . . . , 𝑛} ∀∀ 𝑖1 < 𝑖2 < . . . < 𝑖𝑘 𝑃{𝐴𝑖1 , . . . , 𝐴𝑖𝑘 } = 𝑃{𝐴𝑖1 } · . . . · 𝑃{𝐴𝑖𝑘 }
Пример. (Бернштейна) 
Рассмотрим правильный тетраэдр, на одной грани которого «написано» 1, второй — 2, третьей — 3, четвёртой — 1, 2, 3.
Этот тетраэдр один раз подбрасывают. 
Событие 𝐴1 заключается в том, что на нижней грани «написано» 1; также введём 𝐴2 для 2, 𝐴3 для 3. Давайте покажем, что события 𝐴1, 𝐴2, 𝐴3 попарно независимы, но не являются независимыми в совокупности. 
1. Докажем, что они независимы попарно. Т. к. 𝑃(𝐴1) = 1 2 , 𝑃(𝐴2) = 1 2 , то
𝑃(𝐴1𝐴2) = 𝑃(𝐴1) 𝑃(𝐴2) = ¼
Событие 𝐴1𝐴2 означает, что на нижней грани присутствуют и 1, и 2. Всё аналогично для 𝑃(𝐴1𝐴3) = 𝑃(𝐴1)𝑃(𝐴3) и 𝑃(𝐴2𝐴3) = 𝑃(𝐴2)𝑃(𝐴3).
2. Проверим равенство 𝑃(𝐴1𝐴2𝐴3) = 𝑃(𝐴1) 𝑃(𝐴2) 𝑃(𝐴3), которое, казалось бы, должно равняться 1/8 . Но произведение событий 𝐴1, 𝐴2, 𝐴3 означает, что на нижней грани присутствуют и 1, и 2, и 3, вероятность чего равна 1/4 . И выходит, что 1/4 ̸= 1/8 .
Следовательно, события 𝐴1, 𝐴2, 𝐴3 не являются независимыми в совокупности.

\subsection{Доказать теорему о формуле полной вероятности.}

Пусть Ω — пространство элементарных исходов, связанных с некоторым случайным экспериментом, а (Ω, 𝛽, 𝑃) — вероятностное пространство этого случайного эксперимента.
Определение. Говорят. что события 𝐻1, . . . , 𝐻𝑛 ∈ 𝛽 образуют полную группу событий, если 
1. 𝑃(𝐻𝑖) > 0, 𝑖 = 1, 𝑛;
2. 𝐻𝑖𝐻𝑗 = ∅ при 𝑖 ̸= 𝑗; 
3. 𝐻1 + . . . + 𝐻𝑛 = Ω.
Теорема. Формула полной вероятности. Пусть 
1. 𝐻1, . . . , 𝐻𝑛 — полная группа событий; 
2. 𝐴 ∈ 𝛽 — событие. 
Тогда (это выражение называется формулой полной вероятности):
𝑃(𝐴) = 𝑃(𝐴 | 𝐻1)𝑃(𝐻1) + . . . + 𝑃(𝐴 | 𝐻𝑛)𝑃(𝐻𝑛)
Доказательство:

\subsection{Доказать теорему о формуле Байеса.}

Теорема

\subsection{Доказать формулу для вычисления вероятности осуществления ровно k успехов в серии из n испытаний по схеме Бернулли..}
