\section{Рубежный контроль 2}

\subsection{Сформулировать определение несовместных событий. Как связаны свойства несовместности и независимости событий?}

\textbf{Определение}. 

События $A$ и $B$ называются несовместными, если их произведение пусто. В противном случае события $A$ и $B$ называются совместными.

\textbf{Определение}. 

События $A$ и $B$ называются независимыми, если $P(AB) = P(A) \cdot P(B)$.

\textbf{Как связаны}

Если события несовместные, то они не могут быть независимыми.

\subsection{Сформулировать геометрическое определение вероятности.}

Геометрическое определение вероятности является обобщением классического определения на случай, когда $|\Omega| = \infty$. 

\textbf{Пусть}

\begin{enumerate}
	\item $|\Omega|$ $\subseteq$ $R^n$;
	\item $\mu(\Omega|$ < $\infty$, где $\mu$ --- некая мера. Если $n$ = 1, то $\mu$ --- это длина; если n = 2, то $\mu$ --- площадь; если n = 3 --- объём. Можно определить меры и при больших n; 
	\item Возможность принадлежности некоторого элементарного исхода случайного эксперимента событию $ A \subseteq$ $\Omega$ пропорциональна мере этого события и не зависит от формы события A и его расположения внутри $\Omega$. 
	
\end{enumerate} 

Тогда \textbf{Определение}. 

Вероятностью случайного события $ A \subseteq$ $\Omega$ называют число

\begin{equation}
	P(A) = \frac{\mu(A)}{\mu(\Omega)}
\end{equation} 

\subsection{Сформулировать определение сигма-алгебры событий. Сформулировать ее основные свойства.}

Для строгого аксиоматического определения вероятности необходимо уточнить понятие события:

\begin{enumerate}
	\item Данное выше определение события как произвольного подмножества множества $\Omega$ в случае бесконечного множества $\Omega$ приводит к противоречивой теории (см. парадокс Рассела);
	\item Таким образом, необходимо в качестве события рассматривать не все возможные подмножества множества $\Omega$, а лишь \textbf{некоторые из них}; 
	\item Набор подмножеств множества $\Omega$, выбранных в качестве событий, должен обладать рядом свойств. Понятно, что если A и B --- связанные со случайным экспериментом события и известно, что в результате эксперимента они произошли (или не произошли), то естественно знать, произошли ли события $A + B, A \cdot B, \dots$
\end{enumerate}

Эти соображения приводят к следующему определению.

\textbf{Пусть} 

\begin{enumerate}
	\item $\Omega$ --- пространство элементарных исходов, связанных с некоторым случайным экспериментом; 
	\item $\beta \neq \varnothing$ --- система (набор) подмножеств в множестве $\Omega$.
\end{enumerate}


\textbf{Определение}. 

$\beta$ называется сигма-алгеброй событий, если выполнены условия: 
\begin{enumerate}
	\item если $A \in \beta$, то $\overline{A} \in \beta$;
	\item если $A_1, \dots, A_n, \dots \in \beta$, то $A_1 + \dots + A_n + \dots \in \beta$
\end{enumerate}

\textbf{Свойства}

\begin{enumerate}
	
	\item $\Omega \in \beta$;
	\item $\varnothing \in \beta$;
	\item если $A_1, \dots\ A_n, \dots\ \in \beta$, то $A_1 \cdot A_2 \cdot \dots\ A_n \dots\ \in \beta$;
	\item если $A_1, A_2 \in \beta$, то $A_1 \backslash A_2 \in \beta$
\end{enumerate}


\subsection{Сформулировать аксиоматическое определение вероятности. Сформулировать основные свойства вероятности.}

\textbf{Пусть} 

\begin{enumerate}
	\item $\Omega$ --- пространство элементарных исходов случайного эксперимента;
	\item $\beta$ --- сигма-алгебра, заданная на $\Omega$.
\end{enumerate}

\textbf{Определение}

Вероятностью (вероятностной мерой) называется функция $P: \beta \rightarrow R$

\textbf{Аксиомы}

\begin{enumerate}
	\item $\forall A \in \beta => P(A) \geqslant 0$; (аксиома неотрицательности);
	\item $P(\Omega) = 1$ (аксиома нормированности);
	\item Если $A_1, \dots, A_n, \dots$ --- попарно несовместные события, то вероятность осуществления их суммы равна сумме вероятностей осуществления каждого из них по отдельности: $P(A_1 + \dots\ + A_n + \dots\ ) = P(A_1) + \dots\ + P(A_n) + \dots$ (расширенная аксиома сложения). 
\end{enumerate}

\textbf{Свойства}

\begin{enumerate}
	\item $P(\overline{A}) = 1 - P(A)$;
	\item $P(\varnothing) = 0$;
	\item Если $A \subseteq B$, то $P(A) \leqslant P(B)$;
	\item $\forall A \in \beta: 0 \leqslant P(a) \leqslant 1$;
	\item $P(A + B) = P(A) + P(B) - P(AB)$, где A, B $\in \beta$;
	\item Для любого конечного набора событий $A_1, \dots, A_n$ верно
	\begin{equation}
		P(A_1 + \dots\ + A_n) = + \sum\limits_{1 \leqslant i_1 \leqslant n}P(A_{i_1}) - \sum\limits_{1 \leqslant i_1 \leqslant i_2 \leqslant n}P(A_{i_1}, A_{i_2}) + \sum\limits_{1 \leqslant i_1 \leqslant i_2 \leqslant i_3 \leqslant n}P(A_{i_1}, A_{i_2}, A_{i_3}) - \dots
	\end{equation}
\end{enumerate}

\subsection{Записать аксиому сложения вероятностей, расширенную аксиому сложения вероятностей и аксиому непрерывности вероятности. Как они связаны между собой?}

\textbf{Аксиома сложения}

Сложение --- для $\forall$ конечного набора попарно несовместных событий $A_1, \dots\, A_n$ вероятность осуществления их суммы равна сумме вероятностей каждого из них по отдельности: $P(A_1 + \dots\ + A_n) = P(A_1) + \dots\ + P(A_n)$.

\textbf{Расширенная Аксиома сложения}

Если $A_1, \dots, A_n, \dots$ --- попарно несовместные события, то вероятность осуществления их суммы равна сумме вероятностей осуществления каждого из них по отдельности:  $P(A_1 + \dots\ + A_n + \dots\ ) = P(A_1) + \dots\ + P(A_n) + \dots$.

\textbf{Непрерывность}

Для любой неубывающей последовательности событий $A_1 \subseteq A_2 \subseteq \dots \subseteq A_n \subseteq \dots$ и события $ A = \bigcup\limits_i A_i$ верно 
\begin{equation}
	P(A) = \lim\limits_{i \rightarrow \infty} P(A_i).
\end{equation}

\textbf{Связанность}

Из аксиомы сложения и непрерывности следует расширенная аксиома сложения.

\subsection{Сформулировать определение условной вероятности и ее основные свойства.}

\textbf{Пусть} 

\begin{enumerate}
	\item $A$ и $B$ --- два события, связанные с одним случайным экспериментом;
	\item дополнительно известно, что в результате произошло событие B и P(B) > 0.
\end{enumerate}

Условной вероятностью осуществления события A при условии, что произошло событие B, называется число

\begin{equation}
	P(A|B) = \frac{P(AB)}{P(B)}, P(B) \neq 0.
\end{equation}

\textbf{Свойства}

\begin{enumerate}
	\item P(A|B) $\geqslant$ 0;
	\item $P(\Omega|B) = 1$;
	\item $P(A_1 + \dots\ + A_n + \dots\ | B) = P(A_1|B) + \dots\ + P(A_n|B) + \dots$.
\end{enumerate}

\subsection{Сформулировать теоремы о формулах умножения вероятностей для двух событий и для произвольного числа событий.}

\textbf{Теорема}. Формула умножения вероятностей для двух событий 

\textbf{Пусть}

\begin{enumerate}
	\item A, B --- события;
	\item P(A) > 0.
\end{enumerate}

Тогда P(AB) = P(A) P(B|A)

\textbf{Теорема} Формула умножения вероятностей для n событий

\begin{enumerate}
	\item $A_1, \dots, A_n$ --- события;
	\item $P(A_1 \cdot \dots\ \cdot > 0)$.
\end{enumerate}
Тогда 
\begin{equation}
	P(A_1 \cdot A_2 \cdot \dots\ \cdot A_n)  = P(A_1) P(A_2 | A_1)P(A_3|A_1A_2) \cdot \dots\ \cdot P(A_n|A_1 \cdot \dots\ \cdot  A_{n-1}).
\end{equation}

\subsection{Сформулировать определение пары независимых событий. Как независимость двух событий связана с условными вероятностями их осуществления?}

\textbf{Пусть}

A и B --- два события, связанные с некоторым случайным экспериментом.

\textbf{Определение}

События A и B называется независимыми, если
\begin{equation}
	P(AB) = P(A) P(B).
\end{equation} 

\textbf{Замечание}

Разумеется, в качестве определения независимых событий логично было бы использовать условия P(A|B) = P(A) или P(B|A) = P(B). Однако эти условия имеют смысл лишь тогда, когда P(A) или P(B) отличны от нуля. Условие же $P(AB) = P(A) \cdot P(B)$ <<работает>> всегда.

\subsection{Сформулировать определение попарно независимых событий и событий, независимых в совокупности. Как эти свойства связаны между собой?}

\textbf{Определение}

События $A_1, \dots A_n$ называется попарно независимыми, если
\begin{equation}
	\forall \forall i \neq j; i,j \in \{1, \dots, n\} P(A_i A_j) = P(A_i)P(A_j)
\end{equation}


\textbf{Определение}

События $A_1, \dots A_n$ называется попарно независимыми в совокупности, если

\begin{equation}
	\forall k \in \{2, \dots, n\} \forall \forall i_1 < i_2 < \dots < i_k: P(A_{i_1}, \dots, A_{i_k}) = P(A_{i_1}) \cdot \dots P(A_{i_k})
\end{equation}

\textbf{Как связаны}

Из совокупности следует попарность.


\subsection{Сформулировать определение полной группы событий. Верно ли, что некоторые события из полной группы могут быть независимыми?}

Пусть $\Omega$ --- пространство элементарных исходов, связанных с некоторым случайным экспериментом, а $(\Omega, \beta, P)$ --- вероятностное пространство этого случайного эксперимента.

\textbf{Определение}

События $H_1, \dots, H_n \in \beta$ образуют полную группу событий, если

\begin{enumerate}
	\item $P(H_i) > 0, i = 1, n$;
	\item $H_iH_j = \varnothing$ при $i \neq j$;
	\item $H_1 + \dots H_n = \Omega$.
\end{enumerate}

\textbf{События из полной группы могут быть независимыми?}

Нет, не могут быть, так как они несовместные.

\subsection{Сформулировать теорему о формуле полной вероятности.}

\textbf{Теорема} Формула полной вероятности

\textbf{Пусть}

\begin{enumerate}
	\item $H_1, \dots, H_n$ --- полная группа событий
	\item $A \in \beta$ --- событие.
\end{enumerate}

Тогда (это выражение называется формулой полной вероятности):

\begin{equation}
	P(A) = P(A|H_1)P(H_1) + \dots\ + P(A|H_n)P(H_n)
\end{equation}

\subsection{Сформулировать теорему о формуле Байеса.}

\textbf{Теорема} 

Пусть

\begin{enumerate}
	\item $H_1, \dots, H_n$ --- полная группа событий;
	\item $P(A) > 0$.
\end{enumerate}

Тогда

\begin{equation}
	P(H_i|A) = \frac{P(A|H_i)P(H_i)}{P(A|H_1)P(H_1) + \dots\ + P(A|H_n)P(H_n)}, i = \overline{1, n}.
\end{equation}

\subsection{Дать определение схемы испытаний Бернулли. Записать формулу для вычисления вероятности осуществления ровно k успехов в серии из n испытаний.}

\textbf{Определение}. 

Схемой испытаний Бернули называется серия из однотипных экспериментов указанного вида, в которой отдельные испытания независимы, то есть вероятность реализации успеха в n-ом испытанни не зависит от исходов первого, второго, $\dots$, i-1-ого испытаний.

\textbf{Теорема}

Пусть проводится серия из n испытаний по схеме Бернули с вероятностью успеха p. Тогда $P_n(k)$ есть вероятность того, что в серии из n испытаний произойдет ровно
k успехов:
\begin{equation}
	P_n(k) = C_n^kp^kq^{n-k}
\end{equation}

\subsection{Записать формулы для вычисления вероятности осуществления в серии из n испытаний а) ровно k успехов, б) хотя бы одного успеха, в) от k1 до k2 успехов.}

\begin{enumerate}
	\item $P_n(k) = C_n^kp^kq^{n-k}$;
	\item $P(A) = 1 - P(\overline{A}) = 1 - P_n(0) = 1 - C_0^ip^0q^{n-0} = 1 - q^n$;
	\item $P_n(k_1 \leqslant k \leqslant k_2) = \sum\limits^{k_2}_{i=k_1} C_n^ip^iq^{n-i}$
\end{enumerate}

\subsection{Сформулировать определение элементарного исхода случайного эксперимента и пространства элементарных исходов. Сформулировать классическое определение вероятности. Привести пример.}

\textbf{Определение}. 

Множество $\Omega$ всех исходов данного случайного эксперимента называют пространством элементарных исходов. 

\begin{enumerate}
	\item Каждый элементарный исход является «неделимым», т. е. он не может быть разбит на более «мелкие» исходы;
	\item В результате каждого эксперимента обязательно имеет место ровно один из входящих в  элементарных исходов.
\end{enumerate}

\textbf{Пример}

Из колоды в 36 карт извлекают одну карту.

\begin{equation}
	\Omega = \{6_{\text{пик}, \dots, T_{\text{пик}}, 6_{\text{треф}}, \dots, \dots, T_{\text{червей}}}\}, |\Omega| = 36.
\end{equation} 

Можно определить событие $A = $ {извлечена карта красной масти}, то есть A = $\{6_{\text{бубей}}, \dots, T_{\text{бубей}}, 6_{\text{червей}}, \dots, T_{\text{червей}}  \}$, |A| = 18. Если в результате эксперимента извлечена $6 бубей$, то все событие A целиком наступило.

\textbf{Пусть}

\begin{enumerate}
	\item $\Omega$ --- пространство исходов некоторого случайного эксперимента (|$\Omega$| = N < $\infty$);
	\item по условиям эксперимента нет оснований предпочесть тот или иной элементарный исход остальным (в таком случает говорят, что все элементарные исходы равновозможны);
	\item существует событие $A \subseteq \Omega$, мощность |A| = $N_A$
\end{enumerate}

\textbf{Определение}

Вероятностью осуществления события A называется число 
\begin{equation}
	P(A) = \frac{N_A}{N}.
\end{equation}

\subsection{Сформулировать классическое определение вероятности. Опираясь на него, доказать основные свойства вероятности}

\begin{enumerate}
	\item $\Omega$ --- пространство исходов некоторого случайного эксперимента (|$\Omega$| = N < $\infty$);
	\item по условиям эксперимента нет оснований предпочесть тот или иной элементарный исход остальным (в таком случает говорят, что все элементарные исходы равновозможны);
	\item существует событие $A \subseteq \Omega$, мощность |A| = $N_A$
\end{enumerate}

\textbf{Определение}

Вероятностью осуществления события A называется число 
\begin{equation}
	P(A) = \frac{N_A}{N}.
\end{equation}

Свойства вероятности:

\begin{enumerate}
	\item Вероятность P(A) > 0 (неотрицательна);
	\item $P(\Omega) = 1$;
	\item если A, B --- несовместные события, то $P(A+B) = P(A) + P(B)$.
\end{enumerate}

\textbf{Доказательство}

\begin{enumerate}
	\item Т.к. $N_A \geqslant, N > 0 => $ $P(A) = \frac{N_A}{N} \geqslant 0.$
	\item Принимая во внимание, что $N_{\Omega} = |\Omega| = N$, получается $P(\Omega) = \frac{N_{\Omega}}{N} = \frac{N}{N} = 1$.
	\item  Т.к. $\Omega$ --- конечно, A, B $\subseteq \Omega$, то получается, что A, B конечны. Существует формула $|A + B| = |A| + |B| - |AB|$;
	Т.к A и B --- несовместные, то AB = $\varnothing$, из чего следует, что $N_{a+b} = N_a + B_b$. Таким образом,
	\begin{equation}
		P(A +B) = \frac{N_{a+b}}{N} = \frac{N_{a} + N_{b}}{N} = \frac{N_{a} }{N} + \frac{N_{b} }{N} = P(A) + P(B).
	\end{equation}
\end{enumerate}

\subsection{Сформулировать статистическое определение вероятности. Указать его основные недостатки.}

\begin{enumerate}
	\item Некоторый случайный эксперимент произведен n раз;
	\item при этом некоторые наблюдаемое в этом эксперименте событие A произошло $nA$ раз.
\end{enumerate}

\textbf{Определение}

Вероятностью осуществление события A называют эмпирический (то есть найденный экспериментальный путем) предел: 

\begin{equation}
	P(A) = \lim\limits_{n \rightarrow \infty} = \frac{n_a}{n}.
\end{equation}

У статического определения полным-полно недостатков:

\begin{enumerate}
	\item никакой эксперимент не может быть произведен бесконечное много раз;
	\item с точки современной математики статическое определение является архаизмом, так как не дает достаточно базы для дальнейшего построения теории.
\end{enumerate}

\subsection{Сформулировать определение сигма-алгебры событий. Доказать ее основные свойства.}

Для строгого аксиоматического определения вероятности необходимо уточнить понятие события:

\begin{enumerate}
	\item Данное выше определение события как произвольного подмножества множества $\Omega$ в случае бесконечного множества $\Omega$ приводит к противоречивой теории (см. парадокс Рассела);
	\item Таким образом, необходимо в качестве события рассматривать не все возможные подмножества множества $\Omega$, а лишь \textbf{некоторые из них}; 
	\item Набор подмножеств множества $\Omega$, выбранных в качестве событий, должен обладать рядом свойств. Понятно, что если A и B --- связанные со случайным экспериментом события и известно, что в результате эксперимента они произошли (или не произошли), то естественно знать, произошли ли события $A + B, A \cdot B, \dots$
\end{enumerate}

Эти соображения приводят к следующему определению.

\textbf{Пусть} 

\begin{enumerate}
	\item $\Omega$ --- пространство элементарных исходов, связанных с некоторым случайным экспериментом; 
	\item $\beta \neq \varnothing$ --- система (набор) подмножеств в множестве $\Omega$.
\end{enumerate}


\textbf{Определение}. 

$\beta$ называется сигма-алгеброй событий, если выполнены условия: 
\begin{enumerate}
	\item если $A \in \beta$, то $\overline{A} \in \beta$;
	\item если $A_1, \dots, A_n, \dots \in \beta$, то $A_1 + \dots\ + A_n + \dots\ \in \beta$
\end{enumerate}

\textbf{Свойства}

\begin{enumerate}
	
	\item $\Omega \in \beta$;
	\item $\varnothing \in \beta$;
	\item если $A_1, \dots\ A_n, \dots\ \in \beta$, то $A_1 \cdot A_2 \cdot \dots\ \cdot A_n \dots\  \in \beta$;
	\item если $A_1, A_2 \in \beta$, то $A_1 \backslash A_2 \in \beta$
\end{enumerate}


\textbf{Доказательства}

\begin{enumerate}
	\item По определению $\beta \neq \varnothing \Rightarrow \exists A \subseteq Q: A \in \beta;$ из определения сигма-алгебры (аксиома 1) $\exists (A + \overline{A}) \in \beta;$ т.к $A + \overline{A} = \Omega$, то $\Omega \in \beta$.
	\item Т.к. $\Omega \in \beta$, то, по аксиоме 1, $\overline{\Omega} \in \beta$, а $\overline{\Omega} = \varnothing \Rightarrow \varnothing \in \beta$. 
	\item Из существования событий $A_1, \dots, A_n, \dots \in \beta$ по аксиоме 1 следует, что $\exists \overline{A_1}, \dots, \overline{A_n}, \dots, \in \beta$. По аксиоме 2 следует существование объединения $\exists \overline{A_1} + \dots\ + \overline{A_n} +  \dots\ + \in \beta$, и из аксиомы 1 --- существование дополнение этого объединения: $\overline{\overline{A_1} \cdot \dots\ \cdot \overline{A_n} \dots\ }, \in \beta \Rightarrow^{\text{Де-Морган}} \overline{\overline{A_1}} \cdot \dots\ \cdot \overline{\overline{A_n}} \cdot \dots\  \in \beta$, что тривиально преобразуется в $A_1 \cdot \dots  A_n \cdot \dots\  \in \beta$.
	
	\item Из свойств операций над множествами можно заключить, что A \ B = $A \cdot \overline{B}.$ По аксиоме 1, из B $\in \beta => \overline{B} \in \beta$. По следствию 3, $A, \overline{B} \in \beta => A \cdot \overline{B} \in \beta$, что является утверждением $A \backslash B \in \beta$ 
\end{enumerate}


\subsection{Сформулировать аксиоматическое определение вероятности. Доказать свойства вероятности для дополнения события, для невозможного события, для следствия события.}


\textbf{Пусть} 

\begin{enumerate}
	\item $\Omega$ --- пространство элементарных исходов случайного эксперимента;
	\item $\beta$ --- сигма-алгебра, заданная на $\Omega$.
\end{enumerate}

\textbf{Определение}

Вероятностью (вероятностной мерой) называется функция $P: \beta \rightarrow R$

\begin{enumerate}
	\item $\forall A \in \beta => P(A) \geqslant 0$; (аксиома неотрицательности);
	\item $P(\Omega) = 1$ (аксиома нормированности);
	\item Если $A_1, \dots, A_n, \dots$ --- попарно несовместные события, то вероятность осуществления их суммы равна сумме вероятностей осуществления каждого из них по отдельности: $P(A_1 + \dots + A_n + \dots) = P(A_1) + \dots + P(A_n) + \dots$ (расширенная аксиома сложения). 
\end{enumerate}

\textbf{Свойства с доказательствами}

\begin{enumerate}
	\item По аксиоме 2 сигма-алгебры $\exists A + \overline{A} = \Omega; $ по аксиоме вероятности 2 
	
	$P(\Omega) = 1 = P(A + \overline{A})$; 
	
	по аксиоме вероятности 3 ($A, \overline{A}$ несовместны), 
	
	$P(A + \overline{A}) = P(A) + P(\overline{A}) = 1 \Rightarrow P(\overline{A}) = 1 - P(A)$;
	
	\item $P(\varnothing) = P(\overline{\Omega})$; по свойству 1 $P(\varnothing) = 1 - P(\Omega) = \lgroup \Omega=1$ (по аксиоме 2)$\rgroup = 0$;
	
	\item $A \subseteq B \Rightarrow B = A + (B \backslash A)$ 
	
	Тогда $P(B) = P(A + (B \backslash A)) =$ $\lgroup$ A, B $\backslash$ A несовместны, используем аксиому 3$\rgroup = P(A) = P(B \backslash A) \geqslant P(A) \Rightarrow^{{P(B\backslash A \geqslant 0 \text{по акс. 1})}} P(B) \geqslant P(A)$ .

\end{enumerate}

\subsection{Сформулировать аксиоматическое определение вероятности. Сформулировать свойства вероятности для суммы двух событий и для суммы произвольного числа событий. Доказать первое из этих свойств.}

\textbf{Пусть} 

\begin{enumerate}
	\item $\Omega$ --- пространство элементарных исходов случайного эксперимента;
	\item $\beta$ --- сигма-алгебра, заданная на $\Omega$.
\end{enumerate}

\textbf{Определение}

Вероятностью (вероятностной мерой) называется функция $P: \beta \rightarrow R$

\begin{enumerate}
	\item $\forall A \in \beta => P(A) \geqslant 0$; (аксиома неотрицательности);
	\item $P(\Omega) = 1$ (аксиома нормированности);
	\item Если $A_1, \dots, A_n, \dots$ --- попарно несовместные события, то вероятность осуществления их суммы равна сумме вероятностей осуществления каждого из них по отдельности: $P(A_1 + \dots + A_n + \dots) = P(A_1) + \dots + P(A_n) + \dots$ (расширенная аксиома сложения). 
\end{enumerate}

\textbf{Свойства с доказательствами}

\begin{enumerate}

	\item $\forall A, B: A + B = A + (B \backslash A), $ 
	при этом $A \cdot (B \backslash A) = \varnothing.$
	В соответствии с аксиомой 3,
	\begin{equation}
		\label{stroganov}
		P(A + B) = P(A) + P(B \backslash A)
	\end{equation}
	
	B = AB + (B $\backslash$ A), 
	
	причем (AB)(B $\backslash$ A) = $\varnothing$
	
	По аксиоме 3, имеем $P(B) = P(AB) + P(B \backslash A) \Rightarrow P(B \backslash A) = P(B) - P(AB).$ Подставим результат \ref{stroganov} и получим
	
	\begin{equation}
		P(A + B) = P(A) + P(B) - P(AB), где A, B \in \beta
	\end{equation}
	\item Для любого конечного набора событий $A_1, \dots, A_n$ верно
	\begin{equation}
		P(A_1 + \dots\ + A_n) = + \sum\limits_{1 \leqslant i_1 \leqslant n}P(A_{i_1}) - \sum\limits_{1 \leqslant i_1 \leqslant i_2 \leqslant n}P(A_{i_1}, A_{i_2}) + \sum\limits_{1 \leqslant i_1 \leqslant i_2 \leqslant i_3 \leqslant n}P(A_{i_1}, A_{i_2}, A_{i_3}) - \dots
	\end{equation}
\end{enumerate}

\subsection{Сформулировать определение условной вероятности. Доказать, что она удовлетворяет трем основным свойствам безусловной вероятности.}

\textbf{Пусть} 

\begin{enumerate}
	\item $A$ и $B$ --- два события, связанные с одним случайным экспериментом;
	\item дополнительно известно, что в результате произошло событие B и P(B) > 0.
\end{enumerate}

Условной вероятностью осуществления события A при условии, что произошло B, называется число

\begin{equation}
	P(A|B) = \frac{P(AB)}{P(B)}, P(B) \neq 0.
\end{equation}

\textbf{Свойства}

\begin{enumerate}
	
	\item $P(A|B) = \frac{P(AB)}{P(B)} => P(A|B) \geqslant 0.$
	
	\item $P(\Omega|B) = \frac{P(\Omega)}{P(B)} = \frac{P(B)}{P(B)} = 1.$
	
	\item $P(A_1 + \dots\ + A_n + \dots\ | B) =$ 
	
	$ = \frac{P((A_1 + \dots\ + A_n + \dots\ )B)}{P(B)} = $
	
	$= \frac{1}{P(B)} \cdot P(A_1 B + A_2 B + \cdot + A_n B + \dots) = A_i, A_j$ несовместны, 
	
	$i \neq j; A_i B \subseteq A_i, A_j B \subseteq A_j \Rightarrow (A_i B) \bigcap (A_j B) = \varnothing$, и тогда по аксиоме вероятности 3 
	
	$= \frac{1}{P(B)} \cdot [P(A_1 B) + \dots\ + P(A_n B) + \dots] = $ (ряд) $\frac{P(A_1 B)}{P(B)} + \dots\ + \frac{P(A_n B)}{P(B)} + \dots\ = P(A_1 | B) + \dots\ + P(A_n | B) + \dots\ $.
	
	

\end{enumerate}

\subsection{Доказать теоремы о формулах умножения вероятностей для двух событий и для произвольного числа событий.}

\textbf{Теорема}. Формула умножения вероятностей для двух событий 

\textbf{Пусть}

\begin{enumerate}
	\item A, B --- события;
	\item P(A) > 0.
\end{enumerate}

Тогда P(AB) = P(A) P(B|A)

\textbf{Доказательство}.

Т.к. P(A) > 0, то определена условная вероятность

\begin{equation}
	P(B|A) = \frac{P(AB)}{P(A)}
\end{equation}

из чего напрямую следует

\begin{equation}
	P(AB) = P(A) P(B|A)
\end{equation}

\textbf{Теорема}

Формула умножения вероятностей для n событий

Пусть

\begin{enumerate}
	\item $A_1, \dots\, A_n$ --- события;
	\item $P(A_1 \cdot \dots\ \cdot A_{n-1}) > 0$.
\end{enumerate}

Тогда
\begin{equation}
	P(A_1 \cdot A_2 \cdot \dots\ \cdot A_n) = P(A_1)P(A_2 | A_1)P(A_3|A_1A_2) \cdot \dots\ \cdot P(A_n|A_1 \cdot \dots\ \cdot A_{n-1})
\end{equation}

\textbf{Доказательство}

\begin{enumerate}
	\item Обозначив $k=\overline{1,n-1}$, имеем $A_1 \cdot \dots\ \cdot A_k \supseteq A_1 \cdot \dots\ \cdot A_{n-1}$. 
	По свойству 3 вероятности $P(A_1 \cdot \dots\ \cdot A_k) \geqslant P(A_1 \cdot \dots\ \cdot A_{n-1}) > 0$.
	Следовательно, все условные вероятности, входящие в первую часть доказываемой формулы, определены, и можно задавать условные вероятности по типу $P(A_n | A_1A_2 \dots\ A_{n-1})$, и, следовательно, можно пользоваться формулой умножения вероятностей для двух событий.
	
	\item Последовательно применим формулу умножения вероятностей для двух событий $(P(A_{mf}B_{mf}) = P(A_{mf})P(B_{mf}|A_{mf})):$
	
	$P(\underbrace{A_1 \cdot \dots\ \cdot A_{n-1} \cdot}_{A_{mf_1}} \cdot \underbrace{A_n}_{B_{mf_1}}) = $
	
	$\overbrace{P(\underbrace{A_1 \cdot \dots\ \cdot A_{n-2} }_{A_{mf_2}} \cdot \underbrace{A_{n-1}}_{B_{mf_2}})}^{A_{mf1}} \cdot P(\overbrace{A_n}^{B_{mf_1}} | \overbrace{A_1 \cdot \dots\ \cdot A_{n-1}}^{A_{mf_1}}) = $
	
	$\overbrace{P(\underbrace{A_1 \cdot \dots\ \cdot A_{n-3} \cdot A_{n-2} }_{A_{mf_3}} \cdot \underbrace{A_{n-1}}_{B_{mf_3}})}^{A_{mf_2}} 
	\cdot P(\overbrace{A_{n-1}}^{B_{mf_2}} | \overbrace{A_1 \cdot \dots\ \cdot A_{n-2}}^{A_{mf_2}}) \cdot P(A_n | A_1 \cdot \dots\ \cdot A_{n-1}) = $
	
	$= \dots\ =$
	
	$ = P(A_1)P(A_2|A_1)P(A_3|A_1A_2) \cdot \dots\ \cdot P(A_n|A_1 \dots\ A_{n-1})$
\end{enumerate}

\subsection{Сформулировать определение пары независимых событий. Сформулировать и доказать теорему о связи независимости двух событий с условными вероятностями их осуществления.}

Пусть A и B --- два события, связанные с некоторым случайным экспериментом.

\textbf{Определение}

События A и B называются независимыми, если $P(AB) = P(A) P(B)$.

\textbf{Теорема}

\begin{enumerate}
	\item Пусть P(B) > 0. Утверждение <<A и B --- независимы>> равносильно P(A|B) = P(A);
	\item Пусть P(A) > 0. Утверждение <<A и B --- независимы>> равносильно P(B|A) = P(B);
\end{enumerate}

\textbf{Доказательство}

\begin{enumerate}
	\item Сначала докажем, что если A и B --- независимые, то P(A|B) = P(A).
	По определению независимых событий, P(AB) = P(A)P(B). По определению условной вероятности, 
	\begin{equation}
		P(A|B) = \frac{P(AB)}{P(B)} = \frac{P(A) \cdot P(B)}{P(B)} = P(A)
	\end{equation}

	Теперь докажем обратное
	Пусть $P =(A|B) = P(A)$. 
	
	Докажем, что $P(AB) = P(A)P(B)$
	
	\begin{equation}
		P(AB) \stackrel{\text{по формуле умножения вероятностей}}{=} P(B) \cdot \stackrel{=P(A)}{P(A|B)} = P(B)P(A)
	\end{equation}

	\item Доказательство второго пункта теоремы аналогично.
\end{enumerate}

\subsection{Сформулировать определение попарно независимых событий и событий, независимых в совокупности. Показать на примере, что из первого не следует второе.}

\textbf{Определение}

События $A_1, \dots A_n$ называется попарно независимыми, если
\begin{equation}
	\forall \forall i \neq j; i,j \in \{1, \dots, n\} P(A_i A_j) = P(A_i)P(A_j)
\end{equation}


\textbf{Определение}

События $A_1, \dots A_n$ называется попарно независимыми в совокупности, если

\begin{equation}
	\forall k \in \{2, \dots, n\} \forall \forall i_1 < i_2 < \dots < i_k: P(A_{i_1}, \dots, A_{i_k}) = P(A_{i_1}) \cdot \dots\ \cdot P(A_{i_k})
\end{equation}

\textbf{Пример. (Бернштейна)}

Рассмотрим правильный тетраэдр, на одной грани которого <<написано>> 1, второй --- 2, третьей --- 3, четвёртой --- 1, 2, 3.

Этот тетраэдр один раз подбрасывают.

Событие $A_1$ заключается в том, что на нижней грани <<написано>> 1; также введём $A_2$ для 2, $A_3$ для 3. Давайте покажем, что события $A_1$, $A_2$, $A_3$ попарно независимы, но не являются независимыми в совокупности.

\begin{enumerate}
	\item Докажем, что они независимы попарно. Т.~к. $P(A_1) = \frac{1}{2}, P(A_2) = \frac{1}{2}, $ то $P(A_1A_2) = P(A_1)P(A_2) = \frac{1}{4}$\\
	Событие $A_1A_2$ означает, что на нижней грани присутствуют и 1, и 2. Всё аналогично для $P(A1A3) = P(A_1)P(A_3)$ и $P(A_2A_3) = P(A_2)P(A_3)$.
	\item Проверим равенство $P(A_1A_2A_3) = P(A_1)P(A_2)P(A_3)$, которое, казалось бы, должно равняться $\frac{1}{8}$. Но произведение событий $A_1$, $A_2$, $A_3$ означает, что на
	нижней грани присутствуют и 1, и 2, и 3, вероятность чего равна $\frac{1}{4}$.\\
	И выходит, что $\frac{1}{4} \neq \frac{1}{8}$.\\
	Следовательно, события $A_1$, $A_2$, $A_3$ не являются независимыми в совокупности.
\end{enumerate}

\subsection{Доказать теорему о формуле полной вероятности.}

Пусть $\Omega$ — пространство элементарных исходов, связанных с некоторым случайным экспериментом, а $(\Omega, \beta, P)$ --- вероятностное пространство этого случайного эксперимента.

\textbf{Определение}. Говорят, что события $H_{1}, ..., H_{n} \in \beta$ образуют полную группу событий, если:
\begin{enumerate}
	\item $P(H_{i}) > 0, i = \overline{1, n}$;
	\item $H_{i} H_{j} = \varnothing, \text{ при } i \neq j$;
	\item $H_{1} + ... + H_{n} = \Omega$.
\end{enumerate}

\textbf{Теорема}. Формула полной вероятности. Пусть
\begin{enumerate}
	\item $H_{1}, ..., H_{n}$ --- полная группа событий;
	\item $A \in \beta$ --- событие.
\end{enumerate}

Тогда (это выражение называется формулой полной вероятности):
\begin{equation}
	P(A) = P(A|H_{1})P(H_{1}) + ... + P(A|H_{n})P(H_{n})
\end{equation}

\textbf{Доказательство}
\begin{enumerate}
	\item $A = A \Omega \stackrel{\Omega=H_{1} + ... + H_{n}}{=} A \cdot (H_{1} + ... + H_{n}) = AH_{1} + ... + AH_{n}$. \\
	Принимая $i \neq j: H_{i} \neq \varnothing$, но $(AH_{i}) \subseteq H_{i}, (AH_{j}) \subseteq H_{j} \implies (AH_{i})(AH_{j}) = \varnothing$, т.~е. $AH_{i}$ попарно не пересекаются.
	\item Тогда
	\begin{gather*}
		P(A) = AH_{1} + ... + AH_{n} \stackrel{AH_{i} \text{ попарно не пересекаются}}{=} \\
		= P(AH_{1}) + ... + P(AH_{n}) \stackrel{\text{т.~к. } P(H_{i}) > 0, \text{то } P(AH_{i}) = P(H_{i})P(A|H_{i})}{=}\\
		= P(A|H_{1})P(H_{1}) + ... + P(A|H_{n})P(H_{n})
	\end{gather*}
\end{enumerate}

\subsection{Доказать теорему о формуле Байеса.}

\textbf{Теорема}. Пусть
\begin{enumerate}
	\item $H_{1}, ..., H_{n}$ --- полная группа событий;
	\item $P(A) > 0$.
	
	Тогда
	\begin{equation}
		P(H_{i}|A) = \frac{P(A|H_{i})P(H_{i})}{P(A|H_{1})P(H_{1}) + ... + P(A|H_{n})P(H_{n})}, i = \overline{1, n}
	\end{equation}
\end{enumerate}

\textbf{Доказательство}. 

\begin{gather*}
	P(H_{i}|A) \stackrel{\text{по опр. условной вероятности}}{=} \\
	= \frac{P(AH_{i})}{P(A)} \stackrel{\text{по ф-ле умножения в числителе, полной вер-ти в знаменателе}}{=} \\
	= P(H_{i}|A) = \frac{P(A|H_{i})P(H_{i})}{P(A|H_{1})P(H_{1}) + ... + P(A|H_{n})P(H_{n})}, i = \overline{1, n}
\end{gather*}

\subsection{Доказать формулу для вычисления вероятности осуществления ровно k успехов в серии из n испытаний по схеме Бернулли.}

\textbf{Теорема}. Пусть проводится серия из $n$ испытаний по схеме Бернулли с вероятностью успеха $p$. Тогда $P_{n}(k)$ есть вероятность того, что в серии испытаний произойдет ровно $k$ успехов:

\begin{equation}
	P_{n}(k) = C_{k}^{k} p^{k} q^{n - k}
\end{equation}

\textbf{Доказательство}
\begin{enumerate}
	\item Результат проведения серии из $n$ экспериментов запишем с использованием кортежа $(x_{1}, ... , x_{n})$, где
	\begin{equation}
		x_{i} = 
		\begin{cases}
			1, \text{ если в испытаниях имел место успех;}\\
			0, \text{ если в испытании имела место неудача;}
		\end{cases}
	\end{equation}
	\item Пусть
	\begin{equation}
		A = \{ \text{ в серии из } n \text{ испытаний произошло ровно } k \text{ успехов }\}
	\end{equation}


	Тогда $A$ состоит из кортежей, в которых будет ровно $k$ единиц из $n-k$ нулей.
	
	В событии $A$ будет столько элементарных исходов, сколькими способами можно расставить $k$ единиц по $n$ позициям. Каждая такая расстановка однозначно определяется номерами позиций, в которых будут записаны единицы. В остальные позиции буду записаны нули.
	
	Выбрать $k$ позиций из имеющихся $n$ можно $C_{n}^{k}$ способами. Вероятность каждого отдельного исхода равна произведению вероятностей каждого отдельного
	$x_{i}$, и тогда общая вероятность исхода будет равна $p^{k} q^{n - k}$. 
	
	Все испытания независимы; следовательно, все кортежи из $A$ равновероятны,	и их $C_{k}^{n}$ штук, что означает
	\begin{equation}
		P_{n}(k) = C_{k}^{k} p^{k} q^{n - k}
	\end{equation}
\end{enumerate}