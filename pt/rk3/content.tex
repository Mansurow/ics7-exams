\section{Рубежный контроль 3}

\subsection{Сформулировать определения случайной величины и функции распределения вероятностей случайной величины. Записать основные свойства функции распределения.}

Случайной величиной естественно называть числовую величину, значение которой зависит от того, какой именно \textbf{элементарный исход} произошел в результате эксперимента со случайным исходом.
Множество всех значений, которые случайная величина может принимать, называют \textbf{множеством возможных значений} этой \textbf{случайной величины.}

\textbf{Определение}

Пусть ($\Omega$, $\beta$, P) --- вероятностное пространство.

Случайной величиной называется функция $X : \Omega \rightarrow R$
Такая, что $\forall x \in R$ множество $(\omega : X (\omega) < x) \in \beta$.

\textbf{Определение}

Функцией распределения вероятностной случайной величины X называется отображение $F_X : R \rightarrow R$, определяется правилом $F_X(x) = P\{X < x\}$

\textbf{Свойства}

\begin{enumerate}[label=\arabic*.]
	\item $\lim\limits_{x \rightarrow -\infty}F_X(x) = 0, \lim\limits_{x \rightarrow +\infty}F_X(x) = 1$;
	\item $0 \leq F(x) \leq 1$;
	\item $F(x_1) \leq F(x_2)$, при $x_1 < x_2$ ($F_x$ --- не убывающая функция);
	\item $P\{a \leq X < b\} = F_X(b) - F_X(a)$;
	\item $\lim\limits_{x \rightarrow x_0}F_x(x) = F_X(x_0)$ --- функция распределения непрерывна слева в каждой точке $x_0$.
\end{enumerate}

\subsection{Сформулировать определение дискретной случайной величины; понятие ряда распределения.Сформулировать определение непрерывной случайной величины и функции плотности распределения вероятностей.}

\textbf{Определение}: Дискретная

Случайная величина называется дискретной, если множество ее возможных значений конечно или счетно.

\subsection{Сформулировать определение непрерывной случайной величины. Записать основные свойства функции плотности распределения вероятностей непрерывной случайной величины.}

• ) ( )



4) * + ( ) в точках непрерывности плотности распределения
5) * + для любого наперед заданного . 

\subsection{Сформулировать определения случайной величины и функции распределения вероятностей случайной величины. Записать основные свойства функции распределения.}

• n-мерным случайным вектором называется совокупность СВ
( ) ( ), заданных на одном и том же вероятностном
пространстве ( ). Сами СВ называют копонентами СВектора.
• Функцией распределения n-мерного СВектора (
)
(
) называют функцию, значение которой в точке ( )
равно вероятности совместного осуществления событий *
+ *
+, т.е. (
) *
+.
• Свойства двумерной функции распределения:
1) (
)
2) (
) – неубывающая функция по каждому из аргументов х1 и х2.
3) (
) ( )
4) ( )
5) *
+ (
) (
) (
) (
).
6) (
) – непрерывна слева в любой точке ( )

\subsection{Сформулировать определение дискретного случайного вектора; понятие таблицы распределения двумерного случайного вектора. Сформулировать определения непрерывного случайного вектора и его функции плотности распределения вероятностей.}

• Двумерный случайный вектор (Х,У) называют дискретным, если каждая из случайных величин Х и У является дискретной.
Таблицей распределения двумерного СВектор называют таблицу следующего вида:
в верхней строке перечислены все возможные значения
СВ У; в левом столбце – значения
СВ Х;
на пересечении столбца yj и строки xi находится вероятность {
	
} совместного осуществления событий *
+ {
}.
Также обычно добавляют строку Py и столбец Px:
на пересечении Рх и xi записывается число ; на пересечении Ру и yj записывается
• СВектор (Х1,…Хн) называют непрерывным, если его совместную функцию распределения
(
) можно представить в виде
сходящегося несобственного интеграла (

)





Функцию (
) называют совместной двумерной плотностью распределения СВ Х1…Xn, либо плотностью распределения СВектора
(Х1,…Xn); (
)

( )



\subsection{Сформулировать определения непрерывного случайного вектора и его функции плотности распределения вероятностей. Записать основные свойства функции плотности распределения двумерных случайных векторов.}

• Свойства функции плотности двумерных СВекторов:
1) ( )
2) *

( ) 

\subsection{Сформулировать определение независимых случайных величин. Сформулировать свойства независимых случайных величин. Сформулировать определение попарно независимых случайных величин и случайных величин, независимых в совокупности.}

• СВ Х и У называют независимыми, если совместная функция распределения ( ) является произведением одномерных функций
распределения: ( )
( )
( )
• СВ X1…Xn, заданные на одном вероятностном пространстве, называются независимыми в совокупности, если
(
)


( )
( ).

\subsection{Понятие условного распределения. Доказать формулу для вычисления условного ряда распределения одной компоненты двумерного дискретного случайного вектора при условии, что другая компонента приняла определенное значение. Записать формулу для вычисления условной плотности распределения одной компоненты двумерного непрерывного случайного вектора при условии, что другая компонента приняла определенное значение.}

• Пусть дан двумерный СВектор (Х,У) и известно, что СВ У принимает значение у.
• Пусть (Х,У) – дискретный СВектор; *
+ *
+ {( ) (
	)} {
	
}. Пусть для некоторого j

{
	
}
{( ) (
	)}
{
}



. Условной вероятностью того, что СВ Х примет значение xi при условии что У принимает
значение yj, называется число


; набор вероятностей называется условным распределением СВ Х.
• Пусть (ХУ) – непрерывный СВектор. Условной функцией распределения СВ Х при условии называется отображение
( | )
* | + Условной плотностью распределения СВ Х при условии У=у называется функция
( | )
( )
( )
, где f(x,y) – совместная
плотность распределения СВектора.

\subsection{Сформулировать определение независимых случайных величин. Сформулировать критерий независимости двух случайных величин в терминах условных распределений.}

• Пусть (Х,У) – двумерный случайный вектор. Тогда:


\subsection{Понятие функции случайной величины. Указать способ построения ряда распределения функции дискретной случайной величины. Сформулировать теорему о плотности распределения функции от непрерывной случайной величины.}

• СВ У, которая каждому значению СВ Х ставит в соответствие число ( ), называют скалярной функцией скалярной СВ Х. При этом
сама У также является случайной величиной: если Х – ДСВ, то У – также ДСВ; если Х – НСВ, то У может быть НСВ, ДСВ или СВ смешаного
типа.
• Если Х – ДСВ, то ряд распределения У строится следующим образом – в первой строке записываются значения (
), а во вторую
строку переписываются значения

.
• Теорема: если Х – НСВ с плотностью распределения
( ), - монотонная и непрерывно диффернцируемая скалярная функция, а
– обратная к ), то для СВ ( ) функция распределения
( ) ( ( ))|
( )|.

\subsection{Понятие скалярной функции случайного векторного аргумента. Доказать формулу для нахождения значения функции распределения случайной величины Y , функционально зависящей от случайных величин X1 и X2 .}

• Пусть (Х1, Х2) – СВектор, - скалярная функция. СВ (
) называют скалярной функией случайного вектора.
• Теорема: Пусть (Х1,Х2) – НСВектор и ( ). Тогда


\subsection{Сформулировать и доказать теорему о формуле свертки.}

Теорема: пусть (Х,У) – СВектор, непрерывный и независимый, а




\subsection{Сформулировать определение математического ожидания случайной величины (дискретный и непрерывный случаи). Записать формулы для вычисления математического ожидания функции от случайной величины. Сформулировать свойства математического ожидания. Механический смысл математического ожидания.}



для НСВ.
• Механический смысл мат.ожидания: пусть есть стержень, обладающий «вероятностной массой» и в xi лежит еѐ pi часть. Тогда
математическое ожидание задаѐт x0 – центр тяжести для этого стержня. В случае НСВ, f(x) можно интерпретировать как «плотность»
бесконечного стержня.
Свойства МО:
1) Если Х принимает значение х0 с вероятностью 1 (т.е. не является СВ), то MX=x0.
2) , - , -
3) , -
4) Если Х и У независимые, то , - 

\subsection{Сформулировать определение дисперсии случайной величины. Записать формулы для вычисления дисперсии в дискретном и непрерывном случае. Сформулировать свойства дисперсии. Механический смысл дисперсии.}

• Дисперсией СВ Х называют математическое ожидание квадрата отклонения СВ Х от еѐ среднего значения: , - , -



Механический смысл. Дисперсия представляет собой второй момент центрированной СВ Х:
//коментарий автора: это не икс в
нулевой, это икс с кружочком сверху
• Свойства дисперсии:
1) Если СВ Х принимает всего одно значени С с вероятностью 1, то DC = 0
2) , -

3) ,
- (
)
4) , - , если Х и У – независимые СВ

\subsection{Сформулировать определения начального и центрального моментов случайной величины. Математическое ожидание и дисперсия как моменты. Сформулировать определение квантили и медианы случайной величины.}

• Начальным моментом К-го порядка СВ Х называют математическое ожидание К-й степени этой СВ: ,



.
• Математическое ожидание СВ Х – совпадает с моментом первого порядка. Дисперсия совпадает с центральным моментом 2-го порядка.
• Квантилью СВ Х уровня а называется число , определяемое соотношением *
+ *
+ . Медианой СВ Х
называется еѐ квантиль уровня 0.5.

\subsection{Сформулировать определение ковариации случайных величин. Записать формулы для вычисления ковариации в дискретном и непрерывном случаях. Сформулировать свойства ковариации.}

• Коварацией СВ Х и У называется число ( ) ,( )( )- где m1=MX, m2=MY.
